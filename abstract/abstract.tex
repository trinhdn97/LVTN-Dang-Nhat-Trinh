\cleardoublepage
\addcontentsline{toc}{chapter}{Tóm tắt luận văn}

\begin{abstract}
Thế giới hiện đại ngày càng phụ thuộc nhiều vào Internet, các ứng dụng web cũng ngày càng phổ biến và đa dạng hơn về hình thức lẫn chức năng và ngày càng được sử dụng cho nhiều dịch vụ quan trọng. Tuy nhiên, với bản chất là phần mềm máy tính, các ứng dụng web hoàn toàn có khả năng xuất hiện các lỗ hổng bảo mật, điều đó khiến chúng trở thành những mục tiêu giá trị trong các cuộc tấn công bảo mật. Các lỗ hổng này có thể bắt nguồn từ lỗi phát sinh và sự thiếu kinh nghiệm trong quá trình lập trình, hoặc do sự thất bại của hệ thống trong việc phòng thủ khỏi những tác vụ, dữ liệu nguy hiểm từ phía người dùng. Mặc dù trong cộng đồng an toàn thông tin trên thế giới đã xuất hiện nhiều công cụ, dịch vụ và khung thức hỗ trợ kiểm thử bảo mật ứng dụng web nhưng chưa có công cụ nào kết hợp đầy đủ các yếu tố miễn phí, mã nguồn mở, giao diện trực quan, dễ sử dụng, tự động kiểm thử nhanh và ổn định nhiều lỗ hổng bảo mật cơ bản. Trong quá trình thực hiện luận văn này, chúng tôi tiến hành khảo sát về một số phương pháp kiểm thử bảo mật, các thành phần quan trọng cũng như một vài lỗ hổng bảo mật thường gặp ở ứng dụng web, từ đó áp dụng và hiện thực thành công 
ứng dụng kiểm thử bảo mật \textbf{webfuzzer} đáp ứng đầy đủ những yêu cầu trên. 
\end{abstract}