\chapter{Tổng kết}
\section{Các kết quả đạt được}
Trong quá trình thực hiện đề tài ``\textit{Xây dựng ứng dụng kiểm thử bảo mật ứng dụng web thông qua thông điệp HTTP}'', tôi đã tìm hiểu được một vài phương pháp kiểm thử bảo mật phần mềm nói chung và ứng dụng web nói riêng, tìm hiểu về các thành phần quan trọng về khía cạnh bảo mật của ứng dụng web cùng với xu hướng tấn công các ứng dụng web hiện nay. Hơn nữa tôi cũng tìm hiểu được một số lớp lỗ hổng bảo mật phổ biến trên ứng dụng web. Từ những hiểu biết đó, chúng tôi xây dựng thành công một phương pháp mới để hiện thực \textbf{webfuzzer}, một ứng dụng kiểm thử bảo mật ứng dụng web nhanh và ổn định, cho phép người dùng tập trung vào chính xác vào những chi tiết mà họ muốn kiểm thử. Phương pháp sử dụng một phần mở rộng của Burp Suite để chọn trường hợp kiểm thử như của \textbf{webfuzzer} rất hiệu quả và chưa từng được công bố công khai trong cộng đồng. Bên cạnh đó, ứng dụng cũng hiện thực một số phương pháp riêng trong việc chỉnh sửa payload và phát hiện lỗ hổng để tối thiểu hóa tỉ lệ dương tính giả của ứng dụng. Thông qua các tiêu chí đánh giá và kết quả kiểm thử với một số ứng dụng web mục tiêu, chúng tôi nhận định đã hiện thực khá tốt ứng dụng \textbf{webfuzzer}, thỏa mãn những mục tiêu đã đề ra ban đầu.

\section{Hướng nghiên cứu và phát triển trong tương lai}
Trong thời gian sắp tới, tôi có thể thực hiện những công việc sau đây để hoàn thiện hơn ứng dụng về cả chức năng lẫn hiệu năng.
\begin{itemize}
    \item Phát triển mô-đun xác nhận lỗi và các mô-đun liên quan để phát hiện các lỗ hổng khác trên ứng dụng web như các dạng khác của \acrshort{sqli}, remote file inclusion, \acrshort{xsrf},...
    \item Cập nhật thêm các kĩ thuật bypass đã, và sẽ xuất hiện trong tương lai vào ứng dụng. Những kĩ thuật này bao gồm: các phương pháp mới để phát hiện những lỗ hổng hiện tại, phát hiện những lỗ hổng có thể phát sinh từ lỗ hổng hiện tại đó; các kĩ thuật làm rối cộng với payload đặc thù mới để xuyên qua các kĩ thuật chống bypass, các bộ lọc dữ liệu đầu vào của tường lửa ứng dụng web,...
    \item Hiện thực mô-đun do thám (reconnaissance) giúp phát hiện và lưu trữ thông tin về phiên bản các khung thức, \acrshort{cms}, ứng dụng, thư viện mã nguồn,... mà ứng dụng web đó sử dụng, cộng với thông tin về hosting ảo nếu có (trang web này có được host chung với trang web nào khác không, mức độ bảo mật của các trang web đó ra sao,...). Việc này giúp phát hiện nhanh những ứng dụng web nào có thể bị ảnh hưởng bởi những \acrshort{cve} mới được công bố, đồng thời chỉ ra được những nguy cơ bảo mật có thể có ở không chỉ ở tầng ứng dụng mà còn ở tầng mạng của hệ thống web.
    \item Mở rộng mô-đun quản lý nhật kí hoạt động. Mô-đun này sẽ không chỉ lưu lại những điểm cuối (endpoint) có lỗ hổng trong ứng dụng web kèm theo trường hợp kiểm thử phát hiện ra lỗi đó mà còn lưu cả thông tin phiên bản của các ứng dụng mà ứng dụng web đó dùng (như đã liệt kê ở ý trước) để phục vụ cho việc phát hiện các nguy cơ bảo mật trong tương lai.
    \item Hiện thực mô-đun phát hiện bị chặn hoặc nghi ngờ bị lọc mất gói tin bởi \acrshort{ids} được trang bị trên ứng dụng web mục tiêu thông qua \acrshort{http} response sau đó thử triển khai trên \acrshort{vps}.
    \item Tham khảo và áp dụng thêm một số thuật toán tự nghiệm (heuristic) vào các mô-đun xác nhận lỗi, làm rối, kiểm thử trên front-end, socket listener của máy chủ,... để tăng hiệu năng, tỉ lệ phát hiện lỗ hổng và giảm thời gian thực thi của ứng dụng.
\end{itemize}