\chapter{Tổng kết}
\section{Các kết quả đạt được}
Trong quá trình thực hiện đề tài ``\textit{Xây dựng ứng dụng kiểm thử bảo mật ứng dụng web thông qua thông điệp HTTP}'', tôi đã tìm hiểu được một vài phương pháp kiểm thử bảo mật phần mềm nói chung và ứng dụng web nói riêng, tìm hiểu về các thành phần quan trọng về khía cạnh bảo mật của ứng dụng web cùng với xu hướng tấn công các ứng dụng web hiện nay. Hơn nữa tôi cũng tìm hiểu được một số lớp lỗ hổng bảo mật phổ biến trên ứng dụng web. Từ những hiểu biết đó, chúng tôi xây dựng thành công một phương pháp mới để hiện thực \textbf{webfuzzer}, một ứng dụng kiểm thử bảo mật ứng dụng web nhanh và ổn định, cho phép người dùng tập trung vào chính xác vào những chi tiết mà họ muốn kiểm thử. Thông qua các tiêu chí đánh giá và kết quả kiểm thử với một số ứng dụng web mục tiêu, chúng tôi nhận định đã hiện thực khá tốt ứng dụng \textbf{webfuzzer}, thỏa mãn những mục tiêu đã đề ra ban đầu.

\section{Hướng nghiên cứu và phát triển trong tương lai}
Trong thời gian sắp tới, tôi có thể thực hiện những công việc sau đây để hoàn thiện hơn ứng dụng về cả chức năng lẫn hiệu năng.
\begin{itemize}
    \item Xây dựng cơ chế xác thực người dùng (authentication) trên ứng dụng webfuzzer lẫn phần mở rộng Burp Suite.
    \item Phát triển cấu hình kiểm thử và các mô-đun liên quan để phát hiện các lỗ hổng khác trên ứng dụng web như các dạng khác của \acrshort{sqli}, remote file inclusion, \acrshort{xsrf},...
    \item Cập nhật thêm payload, các kĩ thuật bypass đã và sẽ xuất hiện trong tương lai vào ứng dụng. Những kĩ thuật này bao gồm: các phương pháp mới để phát hiện những lỗ hổng hiện tại, phát hiện những lỗ hổng có thể phát sinh từ lỗ hổng hiện tại đó; các kĩ thuật làm rối cộng với payload đặc thù mới để xuyên qua các kĩ thuật chống bypass, các bộ lọc dữ liệu đầu vào của tường lửa ứng dụng web,...
    \item Hiện thực mô-đun phát hiện \acrshort{http} request bị chặn khi gửi đến ứng dụng web mục tiêu.
    \item Hiện thực trang điều chỉnh biến môi trường backend trên \acrshort{ui} của ứng dụng, cho phép người dùng thay đổi các giá trị này mà không cần can thiệp vào mã nguồn backend.
    \item Cho phép người dùng xem \acrshort{http} response trả về tương tự như giao diện kiểm thử của chức năng \texttt{Intruder}.
\end{itemize}