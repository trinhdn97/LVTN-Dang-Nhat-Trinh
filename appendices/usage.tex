\cleardoublepage
\appendix
\chapter{Hướng dẫn sử dụng phần mở rộng Burp Suite}
Trong trường hợp lần đầu sử dụng ứng dụng, ta cần cài đặt phần mở rộng\footnote{Tham khảo tại https://portswigger.net/support/how-to-install-an-extension-in-burp-suite} \texttt{y4t0g4m1.jar} trong đường dẫn gốc của mã nguồn backend vào công cụ Burp Suite. Giao diện chính của phần mở rộng trên Burp Suite được mô tả như Hình \ref{fig:main-burp-extension-interface} dưới đây, chỉ đơn giản gồm một textbox để thiết lập địa chỉ máy chủ ứng dụng nhận request mẫu. Mặc định giá trị của trường này là \href{http://61.28.235.183:13336}{\texttt{http://61.28.235.183:13336}}, là địa chỉ backend ứng dụng được triển khai trên \acrshort{vps} với cổng mặc định là 13336. \acrshort{ui} của webfuzzer hiện đang được triển khai trên đường dẫn \href{http://61.28.235.183:3000}{\texttt{http://61.28.235.183:3000}} tại thời điểm hoàn thành luận văn.
\begin{figure}[H]
  \centering
    \includegraphics[width=\textwidth,keepaspectratio=true]{images/main-burp-extension-interface.png}
  \caption{Giao diện chính của phần mở rộng trên Burp Suite}
  \label{fig:main-burp-extension-interface}
\end{figure}
Sau khi khởi chạy \acrshort{ui} của ứng dụng webfuzzer và cài đặt phần mở rộng trên Burp Suite, ta tiến hành thiết lập proxy trên Burp Suite và trình duyệt web để bắt request mẫu\footnote{Tham khảo tại https://matthewsetter.com/introduction-to-burp-suite/}. Sau đó ta chuyển request mong muốn đến tab \texttt{Intruder} để tạo và gửi request mẫu đến máy chủ webfuzzer. Hình \ref{fig:send-base-request-1} mô tả quá trình tạo request mẫu bằng cách thêm hoặc bỏ các kí tự ``\texttt{\S}'' bao quanh giá trị của tham số và chọn ``\texttt{Send to y4t0g4m1 webfuzzer}''. Request mẫu này sau đó sẽ hiển thị trên khung danh sách request mẫu ở trang bảng điều khiển của \acrshort{ui} ứng dụng webfuzzer để người dùng tiến hành chọn lỗ hổng và kiểm thử.