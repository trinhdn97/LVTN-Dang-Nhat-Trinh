\cleardoublepage
\addcontentsline{toc}{chapter}{Lời cảm ơn}
\begin{dedication}
Nhìn lại những thăng trầm trong quá trình thực hiện luận văn tốt nghiệp cũng như quãng đường sinh viên sắp kết thúc của mình, tôi cảm thấy thật may mắn khi đã nhận được rất nhiều sự động viên, giúp đỡ từ gia đình, những người thầy, anh em, bạn bè, về cả vật chất lẫn tinh thần trong suốt quá trình học tập ở Đại học Bách khoa TPHCM.\par
Đầu tiên, tôi xin gửi lời cảm ơn chân thành nhất từ đáy lòng đến thầy Nguyễn An Khương và anh Đỗ Đình Huân (Senior Red team Security Engineer - VNG Corporation). Ngoài những lời khuyên đầy kinh nghiệm về mặt chuyên môn, học thuật dành cho tôi, hai người còn dạy tôi về thái độ, tinh thần trách nhiệm cần có của một người đàn ông. Anh Huân còn là người đề ra hướng phát triển ứng dụng, luôn theo sát hỗ trợ tôi trong quá trình hiện thực và sửa đổi ứng dụng dựa vào những kinh nghiệm trong nghề bảo mật của mình. Tôi cũng xin gửi lời cảm ơn đến thầy Nguyễn Cao Đạt và thầy Nguyễn Thanh Tùng. Những góp ý trong giai đoạn ban đầu của hai thầy về phương hướng thiết kế một ứng dụng kiểm thử bảo mật là một phần cơ sở để tôi hoàn thành tốt luận văn tốt nghiệp này.\par
Trong khoảng thời gian bắt đầu nghiên cứu về ngành an toàn thông tin, tôi đã rất may mắn được dìu dắt bởi anh Nguyễn Lê Thành, anh Nguyễn Anh Quỳnh, anh Nguyễn Văn Hòa và bạn Nguyễn Quốc Bảo. Mọi người đã luôn tạo điều kiện cho tôi phát triển bản thân, thông cảm và giúp đỡ tôi rất nhiều trong quá trình nghiên cứu khoa học và làm việc ở VNG Corporation.\par
Bên cạnh đó, không thể không nhắc tới anh Lê Nhật Quang và bạn Nguyễn Văn Thành, những người đồng nghiệp, người anh em đáng tin cậy nhất của tôi. Họ luôn ở bên cạnh động viên, hỗ trợ tôi rất nhiều mỗi khi tôi gặp khó khăn trong công việc chuyên môn cũng như trong cuộc sống cá nhân.\par
Ngoài ra, tôi cũng muốn dành lời cảm ơn đến thầy Nguyễn Hứa Phùng, hai em Võ Vĩ Khang (K2017) và Trần Ngọc Tín (K2016), đang sinh hoạt tại câu lạc bộ An toàn thông tin EFIENS, khoa Khoa học và Kỹ thuật Máy tính, đại học Bách Khoa TPHCM. Những góp ý của thầy và hai em rất thiết thực để luận văn này hoàn thiện hơn, về cả hình thức lẫn nội dung.\par
% \clearpage
Sau cùng, tôi muốn bày tỏ lòng biết ơn sâu sắc nhất đến ba mẹ, những người luôn yêu thương tôi vô điều kiện. Ba mẹ sẽ luôn là nguồn động lực to lớn thôi thúc tôi vượt qua những rào cản của bản thân để đạt được những thành công to lớn hơn trong cuộc đời mình.
\begin{flushright}
    \textit{Tp. Hồ Chí Minh, Tháng 9/2020.}
\end{flushright}
\end{dedication}