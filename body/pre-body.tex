\chapter{Phạm vi và hướng phát triển ứng dụng webfuzzer}
Trong phạm vi luận văn tốt nghiệp này, tôi sẽ áp dụng phương pháp kiểm thử fuzzing dưới dạng kiểm thử hộp đen để hiện thực công cụ \textbf{webfuzzer}. Ngoài khả năng tự động hóa cao, việc áp dụng phương pháp này sẽ đem lại nhiều lợi ích khác.
\begin{itemize}
    \item Linh hoạt trong việc kiểm thử nhiều ứng dụng web sử dụng các công nghệ, mô hình, khung thức phát triển phần mềm khác nhau, cũng như cho phép kiểm thử số lượng lớn ứng dụng web trong một khoảng thời gian ngắn.
    \item Đơn giản hóa và chuẩn hóa việc kiểm thử các lỗ hổng bảo mật trên những ứng dụng web có thiết kế cao cấp và phức tạp. Những ứng dụng này thường có số lượng dòng code có thể lên đến hàng nghìn thậm chí hàng triệu dòng, việc kiểm thử mã nguồn đối với phương pháp hộp trắng hoặc hộp xám sẽ khó khăn hơn nhiều.
    \item Dễ dàng đạt được khả năng tự động hóa cao so với việc quét mã nguồn của ứng dụng. Nhìn chung quá trình phát triển công cụ bằng phương pháp này không yêu cầu kĩ năng lập trình cao hoặc những kĩ thuật nhận diện lỗ hổng quá đặc thù.
    \item Chủ yếu tập trung vào việc kiểm thử chức năng của ứng dụng, không quan tâm đến giao diện, hiệu năng hay trải nghiệm người dùng. Điều này giúp ta có mục tiêu rõ ràng và tiết kiệm công sức cho quá trình thiết kế, cải tiến các trường hợp kiểm thử. 
\end{itemize}
Tương tự như trong lĩnh vực kiểm thử phần mềm, việc áp dụng phương pháp kiểm thử fuzzing dưới dạng kiểm thử hộp đen trong quá trình kiểm thử bảo mật ứng dụng web cũng sẽ có một số hạn chế như sau.
\begin{itemize}
    \item Do không nắm được mã nguồn của ứng dụng web nên chiến thuật tốt nhất là ta chỉ nên tập trung kiểm tra những chỗ thường phát sinh lỗ hổng nhất, hoặc "đoán mò" và vượt qua (bypass) cách thức phòng thủ của ứng dụng bằng việc thử sai và làm rối (tampering) dữ liệu kiểm thử.
    \item Số lượng trường hợp kiểm thử kết hợp với các kĩ thuật bypass đã biết là quá lớn, yêu cầu nhiều kinh nghiệm thực tế của người lập trình để chọn lọc và viết ra được một công cụ nhanh và ổn định, tối ưu hóa tài nguyên máy tính và thời gian thực thi. Ta buộc phải đánh đổi thời gian và tài nguyên đó hoặc chấp nhận kiểm thử trên một tập dữ liệu nhỏ hơn chỉ chứa những trường hợp thường gặp nhất.
    \item Phải có hiểu biết sâu rộng về các lỗ hổng bảo mật, đồng thời thường xuyên cập nhật các trường hợp kiểm thử và kĩ thuật tấn công/phòng thủ mới để bổ sung vào công cụ.
\end{itemize}
Hơn nữa, vấn đề đạo đức nghề nghiệp phải luôn luôn được đặt lên trên hết. Trước và trong quá trình kiểm thử ta phải có thỏa thuận với nhà cung cấp ứng dụng web về mục đích và cách thức tiến hành của mình, hoặc, xác định rõ động cơ của bản thân là để phát hiện những lỗ hổng nguy hiểm và sẽ báo cáo lại với nhà cung cấp để họ vá lỗi, tăng cường bảo mật và bảo vệ quyền lợi của người dùng ứng dụng đó. Bên cạnh đó ta cũng cần phải lưu ý về vấn đề pháp lý của mỗi vùng lãnh thổ địa lý hoặc quy định của những nhà cung cấp khác nhau, cũng như trang bị một số hiểu biết nhất định để bảo vệ danh tính và sự riêng tư của bản thân khi tiến hành kiểm thử số lượng lớn các ứng dụng web xuyên quốc gia.\par

\begin{itemize}
    \item Hiện thực ứng dụng sao cho mô phỏng được những thành phần thường sử dụng nhất trong chức năng \texttt{Intruder} của Burp Suite với kiểu tấn công \texttt{Sniper}. 
    \item Việc chuyển đổi cùng một cấu hình kiểm thử từ thẻ này sang thẻ khác khá phiền phức, nhu cầu định ra một cấu hình kiểm thử mặc định rồi dùng để kiểm thử nhiều request mẫu tương tự nhau là cần thiết và giúp tiết kiệm thời gian hơn.
    \item Dễ dàng triển khai ứng dụng kiểm thử ở các dịch vụ cung cấp máy chủ ảo hóa (virtual private service - \acrshort{vps}. Việc kiểm thử bằng Burp Suite trên máy tính cá nhân (đặc biệt là tính năng \texttt{Intruder}) tiêu tốn khá nhiều tài nguyên, cản trở công việc, dẫn đến nhu cầu tách riêng chức năng này để triển khai trên một máy chủ độc lập. Việc này cũng giúp tránh được nguy cơ địa chỉ IP của máy tính cá nhân bị block theo vùng địa lý hoặc bởi các chính sách kiểm soát lưu lượng truy cập bởi ứng dụng web mục tiêu do gửi quá nhiều request đến trong một khoảng thời gian ngắn, giúp người kỹ sư kịp thời định ra phương hướng khác để tiếp cận mục tiêu.
    \item Cần có một cơ sở dữ liệu để lưu trữ kết quả kiểm thử,
\end{itemize}


% \begin{itemize}
%     \item \textbf{Mô-đun xây dựng \acrshort{http} request} sẽ dựng nên một \acrshort{http} request hoàn chỉnh bằng cách thay thế payload có sẵn đã được làm rối vào những chỗ được ta đánh dấu. Request sau khi được dựng lên xong sẽ được gửi đến một điểm cuối (endpoint) của ứng dụng web mục tiêu đồng thời được truyền vào mô-đun xác nhận lỗi cũng với \acrshort{http} response tương ứng mà ứng dụng web trả về cho máy chủ để kiểm tra và lưu vào nhật kí hoạt động sau này. Mô-đun này được hiện thực bằng một hàng đợi lần lượt gửi đi các request hoàn chỉnh đã được dựng lại và chèn payload vào.
%     \item \textbf{Mô-đun socket listener} được hiện thực bằng socket sẽ liên tục lắng nghe mọi \acrshort{http} request mẫu và response gửi đến nó và chuyển tiếp response đến mô-đun xác nhận lỗi hoặc chuyển tiếp request mẫu đến mô-đun xây dựng \acrshort{http} request. 
% \end{itemize}
% Nhiều payload cùng mục đích được gom thành các lớp, phục vụ cho việc kiểm thử, đánh giá và so sánh kết quả với các công cụ khác về khả năng phát hiện cùng một lỗ hổng cụ thể. Các trường hợp kiểm thử này sau đó được sử dụng trong bộ kiểm thử giao diện ứng dụng web và máy chủ. Nhiều trường hợp kiểm thử cùng mục đích được gom thành các lớp, phục vụ cho việc kiểm thử, đánh giá và so sánh kết quả với các công cụ khác về khả năng phát hiện cùng một lỗ hổng cụ thể. \par
% \textbf{Bộ hậu xử lí và lưu trữ nhật kí hoạt động} có chức năng xác định việc áp dụng trường hợp kiểm thử trên giao diện web hoặc gửi đến điểm cuối của ứng dụng web mục tiêu có khai thác được lỗ hổng (nếu có) như mong muốn của ta không.
% \begin{itemize}
%     \item \textbf{Mô-đun xác nhận lỗi trên \acrshort{http} response} xử lí lần lượt từng response theo trình tự first-in-first-out. Mô-đun này được hiện thực dựa trên các phương pháp phát hiện lỗ hổng \acrshort{xss}, \acrshort{lfi} và time-based \acrshort{sqli} như đã trình bày ở trên.
%     \item \textbf{Mô-đun quản lý nhật kí hoạt động} sẽ ghi lại trường hợp kiểm thử và \acrshort{poc} chỉ ra rằng việc áp dụng trường hợp kiểm thử đó sẽ khai thác được lỗ hổng nào căn cứ vào các điểm bất thường trên \acrshort{http} response. Mô-đun này sẽ ghi lại những trường hợp kiểm thử và kết quả kiểm tra (đã được mô-đun xác nhận lỗi cho là có lỗ hổng - \texttt{[passed]}) và cả những trường hợp bất thường chưa rõ nguyên nhân (\texttt{[skeptical]}) để hậu kiểm bằng tay.
% \end{itemize}
