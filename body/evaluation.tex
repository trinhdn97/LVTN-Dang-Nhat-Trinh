\chapter{Kiểm định và đánh giá kết quả}
Chương này chúng tôi trình bày kết quả sử dụng ứng dụng webfuzzer để kiểm thử một số ứng dụng web và so sánh trực tiếp với quá trình kiểm thử bằng chức năng \texttt{Intruder} của công cụ Burp Suite để đánh giá kết quả đạt được thông qua quá trình thiêt kế và hiện thực ứng dụng.\par
Để quá trình đánh giá kết quả được khách quan, chúng tôi sử dụng tập kiểm thử gồm có các trang web có và không có các lỗ hổng \acrshort{lfi} và time-based \acrshort{sqli}. Chúng tôi đồng thời kiểm thử các trang web này bằng webfuzzer và Burp Suite, so sánh kết quả kiểm thử, các payload phát hiện lỗ hổng nếu có, và thời gian kiểm thử mỗi mục tiêu của hai phần mềm này. Máy tính thực hiện kiểm thử trên cả hai phần mềm sử dụng CPU Intel Core i5-9400F và 8GB DDR4 RAM, trên hệ điều hành Ubuntu 20.04 LTS Focal. Ở chức năng \texttt{Intruder} của Burp Suite chúng tôi thiết lập số lượng luồng (threads) là 1, số lần thử gửi lại request khi thất bại là 0, và không áp dụng các phương pháp encode nào khác trên payload cũng như \acrshort{url} vì trong webfuzzer chúng tôi không hiện thực các chức năng này.\par
\FloatBarrier
\begin{table}[ht]
    \centering
    \caption{Kiểm thử đường dẫn }
    \label{tab:testing-results}
    \begin{tabular}[ht]{lcccc}
        \toprule[1pt]\midrule[0.3pt]
            &{}&\multicolumn{3}{c}{\textbf{Loại lỗ hổng phát hiện được}}\\ \cmidrule{3-5}\textbf{Đường dẫn URL} &\textbf{Loại lỗ hổng tồn tại}&Time-based SQLI&XSS&LFI\\ 
        \midrule
            \texttt{http://testphp.vulnweb.com/}&XSS,&X&X& \\
            \texttt{search.php?test=query}&time-based SQLI& & & \\
            \addlinespace
            \texttt{http://testphp.vulnweb.com/}&LFI& & &(X)\\
            \texttt{showimage.php?file=./pictures/1.jpg}& & & & \\
            \addlinespace
            (Lab) \texttt{https://pentesterlab.com/}&XSS&&X& \\
            \texttt{exercises/xss\_and\_mysql\_file/}& & & & \\
            \addlinespace
        \midrule[0.3pt]\bottomrule[1pt]
    \end{tabular}
\end{table}
\FloatBarrier
